% ============================================================
% PROBLEM FORMULATION SECTION
% ============================================================

\section{Problem Formulation}\label{sec:problem}

We formulate sepsis treatment as a finite-horizon Markov Decision Process (MDP) in the offline reinforcement learning setting, where the goal is to learn an optimal policy from a fixed dataset without further environment interaction. We define interpretability through the Linearly Estimated Gradients (LEG) framework for quantitative feature importance measurement.

% ============================================================
\subsection{MDP Formulation}\label{sec:problem:mdp}

The sepsis treatment MDP is defined by the tuple $\mathcal{M} = (\mathcal{S}, \mathcal{A}, \mathcal{P}, \mathcal{R}, \gamma)$. The \textbf{state space} $\mathcal{S} \subset \mathbb{R}^{46}$ captures patient physiological condition through laboratory values, vital signs, and clinical severity scores as detailed in Section~\ref{sec:methods:env}. The \textbf{action space} $\mathcal{A}$ contains 25 discrete actions representing a $5 \times 5$ grid of IV fluid and vasopressor dosing levels. The \textbf{transition dynamics} $\mathcal{P}(s_{t+1} | s_t, a_t)$ are learned from MIMIC-III data via the gym-sepsis simulator \citep{raghu2017sepsis_drl}. The \textbf{reward function} uses sparse terminal rewards: $\mathcal{R}(s_T, a_T) = +15$ for survival, $-15$ for death, and $0$ for intermediate steps. We use discount factor $\gamma = 0.99$.

A policy $\pi: \mathcal{S} \to \Delta(\mathcal{A})$ maps states to action distributions. The goal is to find $\pi^* = \arg\max_\pi V^\pi(s)$ where the value function is:
\begin{equation}
V^\pi(s) = \mathbb{E}_{\tau \sim \pi, \mathcal{P}} \left[ \sum_{t=0}^{H-1} \gamma^t \mathcal{R}(s_t, a_t) \,\Big|\, s_0 = s \right].
\end{equation}
The action-value function $Q^\pi(s, a)$ represents expected return when taking action $a$ in state $s$ and following $\pi$ thereafter. The optimal policy is derived via $\pi^*(s) = \arg\max_a Q^*(s, a)$.


% ============================================================
\subsection{Offline RL Setting}\label{sec:problem:offline}

In offline RL, the agent learns exclusively from a fixed dataset $\mathcal{D} = \{(s_i, a_i, r_i, s'_i)\}_{i=1}^N$ collected under a behavioral policy, without environment interaction during training \citep{levine2020offline}. The central challenge is \textit{distributional shift}: the learned policy $\pi$ may select out-of-distribution (OOD) actions where Q-value estimates are unreliable due to extrapolation error \citep{fujimoto2019offpolicy}.

Conservative Q-Learning (CQL) \citep{kumar2020cql} addresses this via pessimism, penalizing Q-values for OOD actions:
\begin{equation}
\min_Q \, \mathbb{E}_{(s,a,r,s') \sim \mathcal{D}} \left[ \left( Q(s, a) - \left( r + \gamma \max_{a'} Q(s', a') \right) \right)^2 \right] + \alpha \cdot \mathbb{E}_{s \sim \mathcal{D}} \left[ \log \sum_{a} \exp(Q(s, a)) - \mathbb{E}_{a \sim \pi_\beta} [Q(s, a)] \right],
\end{equation}
where $\alpha > 0$ controls conservatism strength. Behavior Cloning (BC) avoids distributional shift by imitating the behavioral policy via supervised learning: $\pi_{\text{BC}} = \arg\max_\pi \mathbb{E}_{(s, a) \sim \mathcal{D}} [\log \pi(a | s)]$, but cannot improve beyond the behavioral policy's performance.


% ============================================================
\subsection{Interpretability via LEG}\label{sec:problem:interp}

Interpretability quantifies the extent to which clinicians can understand policy decisions—critical for regulatory approval \citep{fda2021ai} and clinical trust \citep{holzinger2017xai_healthcare}. We use Linearly Estimated Gradients (LEG) \citep{greydanus2018leg}, a model-agnostic perturbation method measuring feature importance.

For a policy $\pi$ and state $s$, LEG approximates the saliency (gradient) of the Q-function with respect to each state feature via:
\begin{enumerate}
\item Sample $M = 1000$ perturbations $\delta^{(m)} \sim \mathcal{N}(0, \sigma^2 I)$ with $\sigma = 0.05$
\item Evaluate perturbed Q-values: $\Delta Q^{(m)} = Q(s + \delta^{(m)}, a) - Q(s, a)$
\item Fit linear regression: $\Delta Q^{(m)} \approx \sum_{j=1}^{46} w_j \cdot \delta_j^{(m)}$ via OLS
\item Extract saliency scores: $\text{Saliency}_j(s, a) = w_j$
\end{enumerate}

We quantify interpretability via: (1) \textbf{maximum saliency magnitude} $\max_{s, j} |\text{Saliency}_j(s, a)|$, measuring signal strength; (2) \textbf{saliency range}, capturing feature differentiation; (3) \textbf{clinical coherence}, assessing whether top features align with medical knowledge (e.g., blood pressure, lactate for sepsis). Policies with strong saliency signals ($>10$), large ranges, and high clinical coherence are deemed interpretable.

% End of Problem Formulation section