% ============================================================
% CONCLUSION SECTION
% ============================================================

\section{Conclusion}\label{sec:conclusion}

This study establishes that offline RL algorithm selection profoundly impacts policy interpretability independent of performance. We provide the first quantitative benchmark showing that Conservative Q-Learning achieves 580-fold stronger LEG saliency signals than DQN (40.06 vs. 0.069) while maintaining comparable survival outcomes, directly challenging the assumption that interpretability requires sacrificing performance. All algorithms achieve similar overall survival (94.0--95.0\%), yet CQL outperforms DQN on high-severity patients (88.5\% vs. 84.3\%, SOFA $\geq$ 11)—a clinically meaningful 4.2 percentage point gap. CQL's strong, clinically coherent interpretability patterns satisfy FDA explainability requirements and enable clinician validation, positioning it as the algorithm of choice for clinical AI deployment.

Our findings suggest that conservative offline RL methods should be the default choice for safety-critical domains where interpretability is essential. CQL's conservative penalty term biases the Q-function toward values supported by the training dataset, inheriting the behavioral policy's interpretable structure (threshold-based heuristics). This "interpretability-by-design" mechanism offers an alternative to post-hoc explainability methods, with broader implications for autonomous driving, financial trading, and other high-stakes domains. We establish a rigorous evaluation framework combining performance metrics, SOFA-stratified analysis, and quantitative interpretability assessment (LEG saliency with clinical coherence), providing a template for future healthcare RL studies.

Future work should pursue prospective clinical validation, integrate domain knowledge into CQL training through feature-aware regularization, develop healthcare-specific interpretability metrics (stability, parsimony, actionability), and conduct human-in-the-loop experiments to establish causality between interpretability and clinical decision quality. By prioritizing interpretability alongside performance in algorithm design, we can develop clinical AI systems that earn the trust and adoption of clinicians while improving patient care.

% End of Conclusion section